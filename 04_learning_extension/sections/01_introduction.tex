% Introduction
% Succinct, with no subheadings.

% What events were \textit{key} in the fall of an empire? 
% What turns solidified your victory in that board game? 
% What mutations were most important in the evolution of that behavior? 
% Alas, while these are all valid questions, the sheer number of possibilities means that we will likely never know the answers. 
% That is, unless we are talking about digital evolution, in which we have reached a point where it is computationally feasible to decypher which mutations had the largest effect on the final evolved population. 

%In the natural world, we typically see the fruits of evolution's labor. 
%Even in experimental evolution, we only see

\section{Introduction}

In Chapter \ref{chap:learning_case_studies} I conducted four case studies and analyzed the genetic potentiation of associative learning in the Avida digital evolution system. 
As in the previous chapter, here I use ``genetic potentiation'' to refer to changes in the genetic background that promote the evolution of a particular trait or behavior, in this case associative learning \citep{blountHistoricalContingencyEvolution2008}.
%In the previous chapter, 
We found that potentiation can increase over a short period of time, including a single mutation between parent and offspring. 
While I conducted a deep examination of the observed potentiation dynamics in Chapter \ref{chap:learning_case_studies}, I limited myself to examining only four case study lineages. 
Here, I expand this work by again replaying the evolution of associative learning in Avida, but shifting from detailed case studies to aggregate data by analyzing the potentiation dynamics of 50 lineages. 

My overall goal for this work is to provide a more comprehensive point of comparison for other studies on potentiation dynamics. 
While several previous studies of potentiation dynamics exist, they all analyze the dynamics of only one or at most a few lineages.
By analyzing 50 lineages, here we shift from a few data points of potentiation measurements to full distributions backed by statistically powerful conclusions. %will not have a few data points for particular potentiation measurements, but rather distribution estimates. 
With this work, I aim to accomplish two goals: 
1) to allow us to examine previous studies in a new light, possibly identifying trends in potentiation dynamics across systems, and 
2) to provide a benchmark of potentiation dynamics for future studies to compare against.

There are multiple, but not many, papers that directly or indirectly study potentiation dynamics in ways comparable to this work. 
We are testing the effect of accumulated genetic backgrounds on the subsequent evolution of a particular trait. 
Thus, we can discuss other papers that use analytic replay experiments \citep{blountContingencyDeterminismEvolution2018} to test the potentiation of a trait across different points in an evolved lineage. 
Alternatively, we can leverage mutant studies, where experimental evolution is conducted on organisms with and without a particular set of mutations, but an otherwise identical genetic background. 
However, we can only compare against studies that look at the binary categorical outcome of ``did X trait evolve?'' across replicates; we cannot directly compare to similar studies of continuous phenotypes or of evolvability \citep{woodsSecondorderSelectionEvolvability2011, travisanoExperimentalTestsRoles1995, wagenaarInfluenceChanceHistory2004}. 
In practice, you can turn a continuous variable into a categorical variable (e.g., ``did the replicates evolve a size greater than $1\mu m^{2}$''), however such analysis decisions should be made before experimentation; post-hoc analysis can unintentionally inject our own bias, thus creating a statistical skew. 

We find that, at both 50 lineage steps and single-step levels, potentiation for associative learning can increase substantially. 
Further, single-step increases in potentiation are likely to be driven by a single mutation, though we do see both epistatic and non-epistatic interactions between multiple mutations to alter potentiation. 
Finally, we compare our potentiation dynamics against other works that leverage analytic replay experiments to test the effects of actual accumulated genetic history (\citet{blountHistoricalContingencyEvolution2008}, Chapter \ref{chap:learning_case_studies}). 
Additionally, we compare against works that examine constructed one-step mutants \citep{jochumsenEvolutionAntimicrobialPeptide2016a}, or varying levels of shared history across evolutionary replicates \citep{meyerRepeatabilityContingencyEvolution2012}.
We find that several, but not all, of these other studies have observed similarly large increases in potentiation.
This result provides a baseline point of comparison for future potentiation studies, both \textit{in silico} and \textit{in vitro}.
