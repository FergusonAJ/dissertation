% 210/250 words used
% (250 words max, shorter preferred).
When a new evolutionary dynamic is identified, researchers often struggle to understand its long-term effects on evolutionary outcomes.
Evolutionary prediction is always challenging, as subtle nuances of dynamics can interact in unpredictable ways.
Digital evolution systems, however, provide an empirical alternative to prediction: automated replay experiments can be conducted in large numbers to measure a real distribution of outcomes from a given starting point.
Changes in distributions over time can help us understand the long-term implications of seemingly minor events during evolution.
We apply this technique to ``adaptive momentum'', a new framework that explains how phenomena like selective sweeps can temporarily weaken selection and enhance the likelihood of crossing deleterious fitness valleys.
We show that deleterious mutations along the leading edge of a selective sweep can have an outsized influence on the evolutionary fate of a population.
Indeed, we see that evolutionary potential to cross new deleterious valleys drastically increases during selective sweeps.
Moreover, each valley crossing initiates a new sweep, increasing the potential for further discoveries; this increased potential subsides only once all sweeps have concluded.
While we still have much to learn about both adaptive momentum and the role of history in evolution, this work identifies important evolutionary dynamics at play and hones our tools for further studies.
