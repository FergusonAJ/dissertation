\section{Abstract}
%Indeed, when valley crossing is valuable, evolutionary potential spikes during selective sweeps, often leading to cascades of beneficial mutations; potential subsides again once all sweeps have concluded.
%When a novel piece of fundamental evolutionary theory is published, how can researchers begin to integrate it and understand its role in their various systems? 
%Here, we begin to answer this question for the adaptive momentum effect -- where phenomena like novel beneficial mutations create fitness disequilibrium that temporarily lowers selection pressure and can lead to altered rates of adaptation. 
%We disentangle the role of adaptive momentum in evolutionary potential of spatial populations and illustrate how adaptive momentum can be explained by the historical contingency that it generates. 
%Using a minimalist digital model, we leverage analytic replay experiments to quantify populations' evolutionary potential at every generation, showing exactly what population changes influenced the potential. 
%We find that this viewpoint creates an easily digestible view of adaptive momentum's effects. 
%In our spatial model, we show the vital importance of the leading edge of the ongoing selective sweep in the overall fate of the population. 
%Finally, we extend existing methods of analytic replay experiments and show that not only the genetic makeup of the population, but the organization of organisms in the population, can have dramatic effects on potentiation. 
%While we still have much to learn about both adaptive momentum and the role of history in evolution, this work continues unraveling one of the many evolutionary dynamics at play and hones our tools for further studies. 

% Quick draft of an abstract written for EEB Symposium
% Understanding the role of historical contingency in evolution is vital for both post-hoc analyses of evolved populations and predictions of how populations might evolve in the future.  
% Thankfully, in digital and microbial populations we can ``replay'' evolution from multiple time points to empirically identify key points in the population's history that shifted its long-term evolutionary outcomes. 
% Here, we employ these analytic replay experiments to study the adaptive momentum effect -- where a novel adaptation creates disequilibrium in the population and lowers selection pressure on the adapted subpopulation.
% Using a minimalist digital model, we find that adaptive momentum can dramatically alter the fate of the population. 
% Our replay experiments show that adaptive momentum creates transient windows of increased potential for evolutionary exploration, allowing for the traversal of previously-impassable fitness valleys. 
% We show that, in this spatial model, the evolutionary potential in an adaptive momentum window initially lies in the leading edge of the selective sweep. 
% Further, this work is the first to replay entire population snapshots instead of the previously-used clonal samples, highlighting the importance of population structure and dynamics in long-term evolutionary potential. 
% While we still have much to learn about the role of history in evolution, this work continues unraveling one of the many evolutionary dynamics at play and hones our tools for studying the rest. 
\section{Intro}

% % Intro - historical contingency
% The future of an evolving population is inherently limited by that population's history. 
% These historical contingencies take many forms, including genetic changes, environmental perturbations,  and [blah]. 
% While evolutionary biologists have long been interested in the role that historical contingency plays in evolution, only recently have we been able to empirical test these theories and quantify these effects. 

% % Genetic potentiation
% The genetic background of an evolving population limits which mutations are likely to appear and be selected, and thus affects the trajectory of the evolving population. 
% These genetic changes can influence evolution at multiple timescales. 
% At the shortest timescale, epistatic interactions occur when one mutation affects the fitness of another mutation in a way that defies the baseline expectation. 

%When we consider the ``critical'' mutations in the evolution of an asexual population, some mutations are immediately apparent through means such as a substantial increase in fitness that causes the new genotype to sweep the population.
%Other mutations may have had profound influences on the evolution of the population while lacking any identifying characteristics when they first appear.
%One category of these inconspicuous-yet-critical mutations are those that prime the genetic background such that future mutations can have a dramatic impact. 
%This is often referred to as \textit{genetic potentiation} -- when the accumulated genetic background makes the evolution of a particular trait more likely.


% \subsection{Genetic potentiation, historical contingency, and analytic replay experiments}

% The core of evolutionary biology revolves around either predicting the future evolution of a population or retroactively analyzing why a population evolved in the way it did. 
%Part of the issue with these endeavours is that evolution is not a Markov process that selects the optimal mutation at every time step. 
% However, evolution is a complex, messy process that makes accurate prediction a daunting task, and indeed, some dynamics are opaque even to retroactive analyses. 
% Part of this struggle comes from epistasis, where one mutation changes the effect of a second mutation. 
% In particular, we are interested in genetic potentiation, where epistatic interactions allow for earlier mutations to prime the genetic background such that a particular outcome (a mutation, evolution of a behavior, etc) is more likely to manifest. 
%[ TODO - quick overview of genetic potentiation literature ]

% Evolution is inherently dependent on the stochasticity of random mutations and genetic drift. 
% When combined with the principles of epistasis and genetic potentiation, some chance mutations may have had profound influence on the evolution of the population while lacking any identifying characteristics when they first appear (e.g., no sizable fitness benefit). 
% Researchers are thus interested in historical contingency, studying the long-term evolutionary effects of these chance events. 
% [ TODO - quick overview of historical contingency in evolution ]
% For a review of work done to investigate historical contingency in evolution, see \citep{blountContingencyDeterminismEvolution2018}.

% Of the methods outlined by \cite{blountContingencyDeterminismEvolution2018}, here we are particularly interested in analytic replay experiments. 
% Originally demonstrated in \citep{blountHistoricalContingencyEvolution2008}, analytic replay experiments ``replay the tape of life'' by starting multiple replay replicates from various time points in an evolved population's history. 
% By comparing the evolutionary outcomes between two or more timepoints, we can begin to infer what long-term effects stemmed from the mutations between those timepoints. 
% These experiments trade time and effort for power, providing unparalleled explanatory power but at a considerable cost. 
% [TODO - examples?]
% Examples
% Blount 2008
% Woods 
% Meyer
% Dave Bryson / Art Covert / Gabe Yedid?
% Last year's ALife paper
% Psuedomonas aerigonosa

% \subsection{Adaptive momentum}

% Adaptive momentum is a recently-identified evolutionary phenomenon that describes a reduction in selection pressure on some or all members of a population that arise during periods of fitness disequilibrium [CITE].
% This results in temporary ``momentum windows'' during which typically deleterious mutations act more like ``nearly neutral'' mutations [CITE]. 
% There are several dynamics that can result in the disequilibrium state required for adaptive momentum, such as the discovery of a beneficial mutation, range expansions, and increases in carrying capacity. 
% Regardless of the initiating event, a momentum window will follow and persist until a new equilibrium state is achieved. 
% Adaptive momentum itself is a simple phenomenon, but it has far-reaching implications for evolutionary dynamics. 
% While there remain many questions about the impact of adaptive momentum on natural populations, we show in this work that experimental systems can disentangle some of the effects of adaptive momentum from other evolutionary dynamics. 

\section{Methods}

\section{Results}

\section{Discussion}

\section{Conclusion}