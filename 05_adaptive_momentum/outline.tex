%%%%%%%%%%%%%%%%%%%%%%%%%
%%%%%%%% OUTLINE %%%%%%%%
%%%%%%%%%%%%%%%%%%%%%%%%%

% Notes
    % Have not even mentioned epistasis in results/discusson


% Intro
    % Genetic potentiation
        % Historical contingency plays a huge role in evolution
        % One aspect is how the accumulated genetic background influences further evolution
        % This has been referred to as genetic potentiation
            % Can be short-term or long-term
                % Shortest term is just epistasis
    % Analytic replay experiments
        % Analtyic replays are used to study the role of historical contingency in the evolution of a particular population
        % We usually look at this in terms of genetic potentiation
            % The genetic background changes in a way that promotes the evolution of the behavior we care about
    % Adaptive momentum
        % Selection pressure is lowered on the leading edge during a selective sweep
        % This allows us to cross valleys
        % This can be on the same trait or a different trait
% Methods
    % Evolution system
        % Fitness landscape
            % Extremeley simple sawtooth environment
                % Name the peaks so we can refer to them later
        % Population structure
        % Roulette selection
        % Mutation rate
    % Experimental design
        % Phase 1 - Validating our experiment
            % Experiment 1 - Can valleys be crossed?
            % Experiment 2 - Disequilibrium verification
        % Phase 2 - Benchmarking
            % Experiment 3 - Basic benchmarking
            % Experiment 4 - Shuffled benchmarking
            % (?) Experiment 5 - No disequilibrium benchmarking
        % Phase 3 - Replays
            % Experiment 6 - Standard replays
                % We did replays via population snapshots
                % Every 4 updates
            % Experiment 7 - Shuffled replays
                % Take that snapshot and shuffle it each time
    % Data and software availability
% Results
    % Experiment 1 - Can valleys be crossed?
        % Valleys can be crossed
            % We see that double crossing are more common than expected
                % These almost always occur very early on
    % Experiment 2 - Disequilibrium experiment
        % We see that disequilibrium is vital
    % Experiments 3 - Normal benchmarking
        % We see that we lose potential as the sweep progresses
            % Unless
    % Experiment 4 - Shuffled benchmarking
        % We see that potential falls much faster here
    % Experiment 6 - Standard replays
        % We see that potentiation jumps occasionally, and typically aligns to our benchmarking expectations
        % This includes for the second crossing
    % Experiment 7 - Shuffled replays
        % This destroys the potentiation, typically

% Discussion
    % The relaxed selection caused by disequilibrium (i.e., adaptive momentum) can drastically alter a population's fate
        % If we're not in a momentum window, crossing is pure chance
            % In replicates that did not cross, maximum potentiation was 0.7%
            % In replicates that did cross, potentiation rapidly increased from ~0% to 100%
        % This is especially true at the leading edge of a spatial population. 
            % Here we see that the, at least early in the valley, the leading edge is an excellent predictor of the potential to cross
            % This is not to say that a leading edge is required
                % i.e., the original paper supports well-mixed adaptive momentum
        

    % Population construction/order/structure/etc. affects potentiation
        % Previously, only genetic potentiation had been studied
        % Here we clearly demonstrate that *where* the genotypes are in the population can matter significantly
        % Shuffling the population effectively destroys any potentiation

% Conclusion
        










%%%%%%%%%%%%%%%%%%%%%%%%%
%%%%%%%% OLD OUTLINE %%%%%%%%
%%%%%%%%%%%%%%%%%%%%%%%%%

% Intro
    % Adaptive momentum
        % An adaptation in one trait temporarily lowers selection on other traits during the selective sweep
            % This can allow other traits to cross valleys
    % Potentiation via epistatic interactions
        % Previous work has shown that small, seemingly insignificant mutations can make a world of difference in the long term
            % e.g., a neutral mutation that has an epistatic interaction with another site that cascades into a full blow behvaior change
% Methods
    % Experiment setup
        % Extremely simple sawtooth environment
        % Two types of environments
            % Fixed environment: We initiate N momentum windows
            % Stochastic environment: Number of momentum windows can vary
                % Easy
                % Hard
        % Two selection schemes
            % Elite
                % Easiest to analyze
            % Roulette  (fitness-proportional)
                % More realistic
    % Data and software availability
        % Repo here: xxx

% Results
    % 

% Discussion
    
% Conclusion