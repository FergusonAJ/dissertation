While evolution has created a stunning diversity of complex traits in nature, isolating the details for how a particular trait evolved remains challenging.
Specifically, what were the critical events in evolutionary history that made that trait more or less likely to arise?
We must consider historical contingency, where even small changes, such as an apparently neutral mutation, can have substantial influence on long-term evolutionary outcomes. 
Evolutionary biologists have long been interested in the role that historical contingency plays in evolution, but testing hypotheses of its effects has traditionally been difficult and time consuming, if it is even possible at all. 

Here I leverage the speed and power of digital evolution to experimentally test the role of historical contingency in evolution. 
I start by observing how the evolution of phenotypic plasticity stabilizes future evolutionary dynamics. 
Next, I employ analytic replay experiments to empirically test which mutations in a population's history increased the likelihood that associative learning evolves, first as case studies and then using more statistically powerful experimental approaches. 
I demonstrate that single mutations can drastically increase the odds of learning appearing, shifting it from a rare possibility to a near inevitability, and I find that these "potentiating" mutations exist in all studied lineages. 
Finally, I use potentiating mutations to develop an intuitive view into how adaptive momentum increases evolutionary exploration in populations experiencing disequilibrium. 

We are only beginning to scratch the surface of how historical contingency influences evolution, but digital evolution systems can expedite this process by testing these hypotheses and further refining these techniques for use in natural organisms. 
This work, and those like it, are pivotal in understanding how populations previously evolved, how their accumulated history currently affects them, and how they might evolve far into the 