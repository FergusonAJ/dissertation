
While evolution has created a stunning diversity of complex traits in nature, isolating the details for \textit{how} a particular trait evolved remains challenging.
Specifically, what were the critical events in evolutionary history that made that trait more or less likely to arise? %, or are most traits just flukes? 
We must consider historical contingency, where even small changes, such as an apparently neutral mutation, can have substantial influence on long-term evolutionary outcomes. 
Evolutionary biologists have long been interested in the role that historical contingency plays in evolution, but testing hypotheses of its effects has traditionally been difficult and time consuming, if it is even possible at all. 

Here I leverage the speed and power of digital evolution to experimentally test the role of historical contingency in evolution. 
I start by observing how the evolution of phenotypic plasticity stabilizes future evolutionary dynamics. 
Next, I employ analytic replay experiments to empirically test which mutations in a population's history increased the likelihood that associative learning evolves, first as case studies and then using more statistically powerful experimental approaches. 
I demonstrate that single mutations can drastically increase the odds of learning appearing, shifting it from a rare possibility to a near inevitability, and I find that these ``potentiating'' mutations exist in all studied lineages. 
%Finally, I demonstrate that potentiating mutations identified via replay experiments can highlight the importance of population disequilibrium in the adaptive momentum effect. 
Finally, I use potentiating mutations to develop an intuitive view into how adaptive momentum increases evolutionary exploration in populations experiencing disequilibrium. 

We are only beginning to scratch the surface of how historical contingency influences evolution, but digital evolution systems can expedite this process by testing these hypotheses and further refining these techniques for use in natural organisms. 
This work, and those like it, are pivotal in understanding how populations previously evolved, how their accumulated history currently affects them, and how they might evolve far into the future.
%Continuing to increase our understanding of historical contingency will not only help us understand how life evolved to its current state, but it will also help us predict how populations will continue to evolve. 
%A better understanding of history's role in evolution will help us understand how the organisms around us came to be, and also where they might be going in the future. 


% We can look at any organism around us, including ourselves, and ask how any particular trait of that organism came to be. 
% Indeed, this question can be asked at multiple levels. 
% Perhaps we are interested in the biochemical processes that give rise to neuron activation, or the prior evolutionary stepping stones that have led to echolocation in certain mammals. 
% I focus on one particular question: what were the critical events in the evolution of that trait? 
% We can point to possible candidates, 

% How do traits evolve? 

%%%%%%%%%%%% COMPS ABSTRACT BELOW VVVVVVVV


% % Set the stage that evolution is complicated but it becomes a tiny bit easier if we break it into component parts that we can quantify
% While evolution has created a stunning diversity of complex traits in nature, understanding \textit{how} a particular trait evolved remains a major challenge in evolutionary biology.
% Many dynamics can be at play during evolution, but are often summarized into three factors: adaptation (selective pressures), chance (stochastic events), and history (the genetic starting point for continued evolution).
% %Many dynamics can be at play during evolution, but researchers often boil these down into three key factors: adaptation, chance, and history. 
% %To aid in this endeavour, many researchers adopt the view that evolution is a composite of three factors: adaptation, chance, and history.
% The interplay among these factors can be complex and difficult to disentangle.
% %None of these factors operate in a vacuum; it is the interplay between them that creates the various evolutionary dynamics. 
% By conducting replay experiments on actively evolving populations, however, we can measure the role that each factor played in the evolution of a particular trait.
% Furthermore, these techniques allow us to study genotypes along the history of a linage in order to identify changes not only in phenotypic function, but in evolutionary potential.
% %Further, these studies can not only identify the role that each factor played, they can also identify \textit{when} key aspects of this evolution occurred. 

% % What are we doing, broadly?
% Here I propose to leverage and expand upon these experimental techniques in digital systems.
% I plan to investigate the role that history plays in the evolution of early cognitive behaviors such as associative learning and memory usage in navigation.
% Are there common building blocks whose evolution facilitates more complex behaviors?
% Does the likelihood of a trait evolving increase gradually or in bursts?
% What other elements should we consider that might influence the evolution of a trait?
% To get at these questions, I experimentally measure ``trait potentiation'' as the probability of a trait arising for a given starting population and evolutionary conditions.
% %I focus on how potentiation -- the likelihood a target trait evolves from a given genotype -- changes over the course evolved lineages.

% My use of digital systems allows me to conduct analytic replay experiments on a massive scale.
% Specifically, I can target individual points in evolutionary history and conduct replay experiments to measure their potentiation.
% By comparing potentiation before an after a given mutation, I am able to pinpoint specific mutations that affect potentiation.
% Points that increase potentiation inherently shift the evolution of a trait from requiring chance events to being able to rely on adaptive pressures.
% I will study lineages that ultimately lead to a trait of interest to determine if potentiation increases gradually or if the trait's appearance flips suddenly from fluke to inevitability.

% %switch the evolution of a focal trait from a fluke to an inevitability -- from chance to adaptation.  
% I have two main aims in this dissertation proposal: 
% 1) I want to understand the evolution of potentiation, and more broadly, how history interacts with adaptation and chance to produce complex traits and behaviors, and 
% 2) I want to explore how cognitive behaviors evolve, and more specifically, why these behaviors so rarely arise in digital evolution systems. 
% While these techniques have been refined over the last few decades, here I propose to conduct them on scales only feasible in digital systems.
% As described below, I will fully explore whole lineages, and even full fitness landscapes, while disentangling the effects of different traits, environments, and representations.
% My goal is to develop a more holistic understanding of how potentiation changes during evolution.


% % I will fully explore many successful (and some unsuccessful) lineages, isolating the effects of each mutation and, in case studies, disentangling their interactions with the background genomes.
% % Furthermore, I will conduct potentiation measures for each possible starting point in simple digital systems, to fully understand the potentiation landscape and how evolutionary dynamics interact with it.
% % to allow for the first cross-environment and cross-representation comparisons of how potentiation changes along lineages. 


% % What are we doing, per chapter?
% I start this proposal (Chapter \ref{chap:intro}) with a review of the relevant background in adaptation, chance, and history, as well as my own perspective on how I view these topics.  
% I also provide an overview of prior work on the evolution of cognitive behaviors in digital systems. 
% Next (in Chapter \ref{chap:learning_case_studies}) I demonstrate that single mutations can drastically increase the potentiation of associative learning, and follow that up with a proposal (Chapter \ref{chap:learning_distributions}) to expand the scale of this work to allow for deeper analyses and more powerful statistical comparisons. 
% In Chapter \ref{chap:adaptive_momentum}, I take a step back and examine potentiation in a simplified bitstring model, which I propose as a mechanism to better understand the basics of potentiation and how it relates to epistatic interactions. 
% Thinking about the role of history in evolution more broadly, in Chapter \ref{chap:consequences_of_plasticity} I discuss published investigations into how the evolution of phenotypic plasticity, a common stepping stone for cognitive behaviors, shapes future evolutionary dynamics.
% In order to assess the robustness of earlier results, in Chapter \ref{chap:adaptive_momentum} I propose to compare the associative learning potentiation work to new studies that I will conduct using my phenotypic plasticity environment as well as a new navigational behavior environment.
% Finally, in Chapter \ref{chap:conclusion}, I provide a timeline for this work and identify some additional assessments that should be performed in the future (or, perhaps, as alternatives to the chapters above).  For example, the underlying representations of the digital organisms could be varied to investigate the generalizability of patterns across organism types. 

% Overall, I believe that these studies will help us gain a deep understanding about the evolution of potentiation, with strong implications for the evolution of evolvability and ideally the prediction of evolutionary outcomes.
