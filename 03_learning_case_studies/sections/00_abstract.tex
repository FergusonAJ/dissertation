% 248/250 words used
% (250 words max, shorter preferred).

Scientists have long tried to predict evolutionary outcomes in order to design vaccines for next year's diseases, stabilize endangered ecosystems, or make better choices in designing evolutionary algorithms.
To predict, however, we must first be able to retroactively identify the key steps that determined the evolved state.
Researchers have long examined the role of historical contingency in evolution; when do small, seemingly insignificant mutations substantially shift the probabilities of what traits or behaviors ultimately evolve?
Practitioners of experimental evolution have recently begun to investigate this question using a new technique: analytic replay experiments.
We can found many populations with a given genotype in order to measure the probability of a particular trait evolving from that starting point; we call this the ``potentiation'' of that genotype. 
Moving along a lineage, we can identify which mutations altered potentiation.
Here we used digital organisms to conduct a high-resolution analysis of how individual mutations affected the potentiation of associative learning. %version of this technique.
We find that the probability of evolving associative learning can increase suddenly -- even with a single mutation that appeared innocuous when it occurred. 
While there was no obvious signal to identify potentiating mutations as they arose, we were able to retrospectively identify mechanisms by which these mutations influenced subsequent evolution.
Many of the most interesting and complex evolutionary adaptations that occur in nature are exceptionally rare.
Here, we extend techniques for understanding these rare evolutionary events and the patterns and processes that produce them.
% Here, we extend these techniques for understanding the origins of interesting and rare evolutionary events, as well as the patterns and processes that produce them.


% Scratch

%We isolated how individual mutations altered the likelihood for learning or pre-learning strategies to evolve, with a focus on associative learning.
%We isolated how individual mutations altered the likelihood for associative learning to evolve.
%By taking an evolved lineage and founding new evolving populations from various points along that lineage, we can measure any changes to the likelihood that a certain trait eventually evolves, known as the ``potentiation'' of that trait.
%Due to the stochastic nature of evolution, not only is it hard to predict evolutionary outcomes, it is difficult to look at an evolved lineage and determine the key steps that pushed the population toward its final evolved state. 
%varied over lineages that successfully evolved associative learning in a digital evolution environment. 
%By leveraging a two-step analysis of lineages, we can zoom in to see the effect of individual mutations on potentiation. 
%This work demonstrates the potential of this technique in aiding our understanding of the evolution of intelligence in future studies. 