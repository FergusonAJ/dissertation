Main idea: What do folks need to know to understand this dissertation? 
Main points, according to Chat: 
    - Contextualize the research
    - Establish the research problem
    - Review relevant literature
    - Justify the study
    - Outline the dissertation structure / summarize chapters

Addressing these main points: 
    - Contextualizing the research
        - This where we create a shared vocabulary
            - History in evolution
            - Historical contingency
            - Potentiation and replay experiments
        - Establish the research problem 
            - ???
                - Do I have one singular research problem?
                - Is the overall "What mutations altered the trajectory to lead us here?"
        - Review the relevant literature
            - Do we do this in a separate section, or weave these in throughout the contextualization? 
                - E.g., Travisano for history generally, Blount, Jochmusen, etc for potentiation
        - Justify the study
            - Why do we need to go beyond these other papers? 
                - They have a lot of promise, but we also have a lot of questions left unanswered. 
        - Outline the structure / contributions

- The role of history in evolution
    - Probably easiest to couch this in the adaptation, chance, and history framework
        - History is just the starting point of "continued evolution" 
            - i.e., it depends on our viewpoint
        - The effect of history can be observed as a difference between times A and B
            - In particular, a difference in the roles of chance and adaptation
    - How have people tested this? 

- Genetic potentiation
    - Genetic potentiation is the likelihood of evolving a given outcome given the particular genetic starting point
    - How do we measure this? 
        - Test it with different genetic starting points!

- Historical contingency
    - We know that history matters, and that we can measure potentiation
    - But here we are interested in something more -- historical contingency
        - Broadly speaking, we are interested in how the accumulation of changes (often insignificant in themselves) can drastically alter long-term dynamics

- Thesis statement
    - Experimental evolution allows us to not only quantify immediate changes in fitness, but we can now quantify changes in long-term potential. 
        - This allows us to identify which mutations were critical in what ultimately evolved. 
    - We can leverage digital evolution to provide novel insight into potentiation dynamics and techniques, while also leveraging the techniques to better understand existing evolutionary dynamics. 

- Contributions
    - Chapter 2 (Consequences of plasticity)
        - While many studies focus on the evolution of phenotypic plasticity, here we use digital evolution to identify how evolutionary dynamics change in populations that have evolved phenotypic plasticity
            - We find X, Y, Z. 
    - Chapter 3 (Replaying learning -- case studies)
        - We perform analytic replay experiments to understand the potentiation of associative learning in the digital evolution system Avida. 
            - We find that potentiation can increase suddenly, even with a single mutation
                - We then dive in to uncover the details of these specific potentiating mutations
    - Chapter 4 (Replaying learning -- extension)
        - Here we expand upon Chapter 3, switching from individual case studies to a much larger sample size
            - We identify distributions for key potentiation metrics, such as the largest potentiation gain with a single lineage step
                - We verify that Chapter 3 was not a collection of flukes, and further verify that these large increases are common, but not quite ubiquitous in studies of natural organisms. 
    - Chapter 5 (Replaying adaptive momentum)
        - Adaptive momentum is a new evolutionary dynamic which has support via aggregate data but no studies on the dynamics of populations under adaptive momentum
            - Here we do just that via analytic replay experiments
                - We find that disequilibrium clearly increases potentiation, and every mutation counts under AM, while outside AM it is pure chance
                - We also verify that population structure matters in AM caused by a selective sweep
            - This also demonstrates the difference of population snapshots vs clonal restarts 
    - Chapter 6 (Conclusion)
        - I reflect on my thoughts about what has been done with analytic replays, and propose new studies to really strengthen this technique