\section{Why digital evolution?}

% Traditional methods are limited - Introduce mindset + topic
Researchers addressing these questions about the role of history in evolution initially started where many evolutionary biology questions started: in the fossil record. 
Indeed, some of the most influential thoughts on the role of history come from fossils in the Burgess Shale \citep{gouldWonderfulLifeBurgess1990}.
More recent techniques allow researchers to supplement these ideas with genomic analysis and phylogenetic reconstruction, powerful tools for determining the history of life on earth. 
Looking to history, however, has many limitations: we are provided with only a single instance of evolution, the data we do have is incomplete, and we are not able to go back in time to conduct controlled experiments.
In recent decades there has been a surge in experimental approaches to studying evolution \citep{kaweckiExperimentalEvolution2012}.
If we evolve populations in controlled laboratory conditions, we are able to evolve many populations in parallel, observe almost everything that occurs, and build a range of experimental conditions -- thus addressing each of the problems above.
Evolutionary biologists have leveraged experimental evolution to empirically test a multitude of hypotheses, such as possible selection pressures for the transition to multicellularity \citep{ratcliffExperimentalEvolutionMulticellularity2012} and the feedback cycle between phage- and antibiotic- resistance and its clinical implications \citep{kortrightPhageTherapyRenewed2019, chanPhageSelectionRestores2016}.
I adopt this experimental mindset throughout this dissertation in an attempt to understand the role of historical contingency in evolution. 
I will use the possibilities revealed from counterfactual experiments to understand ``life as it could have been'' in our study systems.
These techniques will help me separate the flukes from the inevitabilties in the dynamics that shaped the course of evolution as it was originally realized in these studies.

However, even the control and speed of \textit{in vitro} experimental evolution is sometimes not enough. 
%The challenge with addressing these problems using standard experimental evolution techniques is that they require us to have speed, control, and data collection capabilities beyond what is currently possible.
Specifically, we must be able to isolate all of the individual mutations along a lineage and replay evolution using each step as a new starting point.
Furthermore, for each of these starting points, we need to be able to conduct enough replays to generate statistically powerful conclusions about evolutionary outcomes.
While experimental evolution techniques are powerful (and getting more powerful with time), performing experiments at this scale would still take decades of dedicated work in the laboratory. 

To overcome these hurdles, I leverage the power of computational evolution systems.
These systems maintain populations of self-replicating computer programs that mutate and evolve in the same way as natural organisms.
In fact, compelling arguments have been made that digital evolution is not a simulation of evolution, but rather an instantiation of it \citep{pennockModelsSimulationsInstantiations2007}.
These digital evolution systems are not a departure from experimental evolution, but rather a choice of experimental system. 
These computational systems give many benefits over \textit{in vitro} approaches, including greater speed, automatic high resolution data collection, and the ability to start an experiment with the exact conditions of our choosing.
Of course, digital evolution also has its drawbacks. 
For example, there is a much lower limit to the complexity of the organisms and what we have been able to evolve \textit{in silico}, and open-ended evolution remains elusive \citep{taylorOpenEndedEvolutionPerspectives2016}.
Further, due to technological constraints, digital evolution is typically limited to scales of at most tens of thousands of organisms, while natural populations can be vastly larger \citep{morenoTrackableAgentbasedEvolution2024}.
%As such, evolving meaningfully complex traits from scratch in open-ended systems requires a better understanding of the underlying dynamics to be able to maximize evolvability. 

As a digital evolution researcher, I am often asked how my computational models tell us anything about how life evolved in the natural world. 
%This notion is a misconception.
While other computational researchers may ask such questions, I do not claim to \textit{directly} explain life \textit{as it was or will be}.
Instead, my questions are theoretical; studying evolution as a process using \textit{life as it could be}.
My goal is to conduct experiments that are infeasible or impossible to conduct in laboratory studies of evolution to further refine our understanding of basic evolutionary dynamics.
%Instead, my questions are grounded in \textit{life as it could be}. 
%However, while not directly studying life as it evolved on earth, we are still able to make claims about how evolution works in the real world. 
For example, we can show evidence of the minimal set of requirements for the evolution of a particular trait or behavior \citep{pontesEvolutionaryOriginAssociative2020, lalejini_evolutionary_2016}. 
We can also boil questions down to their most basic scenarios, removing all possible confounding factors in a way that is not possible in natural organisms and allowing for strong claims about the impacts of the remaining dynamics \citep{wilke_evolution_2001, Bohm2024.04.08.588357}. 
Finally, in the case of history in evolution, we can show examples of how these dynamics unfold, though we cannot claim that is how these dynamics will \textit{always} unfold.
By improving our theoretical understanding and fine-tuning experimental methodologies, we enable wet-lab or natural-system evolutionary biologists to better understand their study system, and, more generally, the evolution of life on earth. 
%Other researchers can then apply our theoretical understanding and methodological improvements to their study systems, 
%Overall, in digital evolution we must walk a fine line in what we claim our research says, and I aim to clarify exactly what the digital studies tell us throughout this dissertation. 

% While there are many complex traits that could be studied, here I focus on the evolution of early cognitive behaviors. 
% This topic is of great interest to evolutionary biologists in understanding the origins of intelligent behaviors.
% The lack of obvious physical characteristics of intelligence makes it challenging to study the evolution of these traits by looking at the fossil record, and their complexity makes them difficult to re-evolve under laboratory conditions.
% Of course, this domain has also proven challenging for evolving digital systems, which is of little surprise as artificial intelligence as a field has encountered many hurdles in its quest to produce intelligent agents. 
% Investigations into the origins of simple stepping stones to intelligence, however, have been much more fruitful. 
% By looking at the role of history in the evolution of early cognitive behaviors, I aim to shed some light on how these behaviors arise, why they are so difficult to evolve, and how we might increase the complexity of intelligent behaviors that digital evolution systems can produce.