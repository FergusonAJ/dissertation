\section{Analyzing evolution: selection, chance, and history}

% Introduce adaptation, chance, and history
In thinking about the evolution of a particular trait, I adopt a mindset modified from \citet{travisanoExperimentalTestsRoles1995} and consider the contributions of three different factors: selection, chance, and history.
New or modified traits that provide a net benefit to the survival and reproduction of an organism (i.e., adaptations) are more likely to increase in frequency, thus illustrating the role that \textit{selection} plays in shaping evolutionary outcomes. 
Here I split from \citet{travisanoExperimentalTestsRoles1995}, focusing on selection as a process, which results in these adaptations becoming more prevalent in the population.
Of course, selection can only act upon genetic sequences that are available in the population.
The stochastic nature of random mutations means that some genetic sequences will arise, while others will never even appear for selection to consider in the first place.
The random appearance of sequences -- or disappearance as misfortune can remove otherwise fit genotypes (i.e., genetic drift) -- highlights some of the roles of \textit{chance} in evolution.
The influence of chance is constrained, though, by the starting genotypes that mutations act upon; traits can appear or be modified only if such changes are available in the local genetic neighborhood.
This distribution of genetic sequences in the population at the point under investigation in an evolutionary study can be described as the product of its \textit{history}.
It is the interplay of these three aspects that produce the complex dynamics that we observe in evolving populations.

% Dive deeper into history
Of course, what we call history is just a matter of temporal perspective.
From the vantage point of a population in a given state, all of the dynamics, including selection and chance, that brought the population to that state are now consolidated under the label of history.
Looking forward, however, both selection and chance will be at play, with different mutations or combinations of mutations occurring at different probabilities, and the resulting combinations having differing survival potential.
As time advances, these changes are again relegated to history, while new outlooks now exist for the population based on its new composition.
As such, for any study where we examine the balance among these three factors, we need to be clear about the starting point from which our perspective will be based.

% Discuss how the three factors interact
%In this work, I focus on the balance between chance and adaptation and how that balance changes over the course of history.
In this work I focus on the role of history in evolution, which is to say that I investigate how the balance of chance and selection shifts over the course of the population's history. 
Since evolution is a stochastic process, we must always discuss possible evolutionary outcomes in terms of probabilities. 
From one point along a lineage, the evolution of a given trait may be incredibly unlikely, dependent on a rare double mutation or a deleterious mutation that is not immediately purified.
From a later point along that lineage, however, that same trait's evolution may shift to being a near certainty, with strong selective pressure for that trait, or its preceding parts, more in control of the population's fate.
Is there some way for us to predict these shifts in influence between selection and chance?
Can we distinguish between a population that is being driven to a specific outcome from one that is simply adrift?
And how do these shifts occur?
Does chance give way to selection in small increments, or can an individual mutation dramatically alter the balance?