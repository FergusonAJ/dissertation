\section{Contributions}

Overall, this dissertation advances our knowledge of the role that history plays in evolution. 
This is not done by inventing new methodologies nor discovering new phenomena. 
Instead, the main contributions of this work lie in the intersections; by applying existing techniques to new areas. 
By leveraging the power of digital evolution, I am able to experimentally test ideas of historical contingency at scales that were previously impossible. 

Moving to specifics, in \textbf{Chapter \ref{chap:consequences_of_plasticity}} I ask a simple question: What happens \textit{after} phenotypic plasticity evolves? 
Here, phenotypic plasticity, where different phenotypes can arise from a single genotype due to environmental variation, is studied because it is a precursor to learning (a theme in later chapters, and what initially brought me to study digital evolution).
However, this question is often difficult; while we can readily find species capable of phenotypic plasticity, what would we compare them to? 
I thus use digital tools to evolve both plastic and non-plastic populations, and then test how their evolutionary dynamics change as a consequence. 
Across all experiments, I find that the evolution of phenotypic plasticity stabilizes future evolution in a cyclic environment that consistently alternates between two states. 
Plastic populations experience less evolutionary change, maintain more novel beneficial traits, and accumulate fewer deleterious traits than their non-plastic counterparts. 
Indeed, evidence suggests that the evolutionary dynamics of plastic populations more closely align to those of populations evolving in a static environment than non-plastic populations in the same environment.  


For the rest of this dissertation, my focus shifts from effect to cause; from identifying the consequences that result from a known change in history to retroactively identifying the changes that shifted the overall evolutionary fate. 
This starts with \textbf{Chapter \ref{chap:learning_case_studies}}, where I conduct case studies to find and characterize ``potentiating mutations'' in the evolution of associative learning. 
By conducting analytic replay experiments \citep{blountContingencyDeterminismEvolution2018}, I identify a single parent-to-offspring reproduction event that substantially increased the likelihood that learning ultimately evolved in each of four case study lineages.
This initial foray demonstrates that even single mutations can drastically alter the fate of an evolving population.

While the results of Chapter \ref{chap:learning_case_studies} enhance our understanding of potentiation and potentiating mutations, the low count of case studies prevents us from being able to make any statistical claims.
Thus we are left with the question: do \textit{most} of the populations that evolve learning in Avida do so via large potentiating mutations, or were those four replicates unlikely? 
To answer this question, I replay 50 additional learning replicates in \textbf{Chapter \ref{chap:learning_distributions}}. 
I find that the four original lineages were not flukes. 
All 50 lineages not only see a significant increase in potentiation across 50 lineages steps, they all see a significant increase in a \textit{single} step. 
Additionally, we see that most of these potentiating lineage steps either occur with a single mutation, or only one mutation had a significant effect on potentiation. 
This work is by far the largest study of potentiation to date, and it allows for comparison against both previous studies and those in the future. 

Finally, \textbf{Chapter \ref{chap:adaptive_momentum}} takes a different approach. 
I leverage analytic replay experiments again, but instead of studying potentiation, I use it as a tool for viewing a newly identified evolutionary dynamic: adaptive momentum \citep{Bohm2024.04.08.588357}, which broadly states that populations in disequilibrium see more evolutionary exploration. 
While adaptive momentum has been shown in the aggregate, here I demonstrate the effect within the history of an evolving population by analyzing the potentiation of crossing a fitness valley.  
I show that when a spatial population is experiencing a selective sweep (an example of disequilibrium), mutations near the leading edge of the sweep have a drastic effect on the overall evolution of the population. 
This contrasts with populations in equilibrium, which effectively rely on pure chance to cross the fitness valley.
Further, I demonstrate an example where the structure of the population matters, as the potentiation of the population diminishes if we shuffle the order of the organisms. 

%The questions about historical contingency that I ask in this dissertation are not new, they have been around for years but were nigh impossible to conduct in either \textit{in vivo} or \textit{in vitro} evolution studies. 
%Instead, the main contributions of this work are
%All together, this work asks questions about historical contingency in evolution that have been around for a while but were either impossible in populations of natural organisms or would require years to conduct \textit{in vitro}.
%I set up a foundation, both of questions and of data, for future studies to build off of and compare against. 
%More than anything, I show that these studies can drastically improve our understanding of what is truly going on as a population evolves, and that this area warrants considerable effort in the future.
