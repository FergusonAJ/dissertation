% General intro - Evolution is amazing but hard to study/understand
I feel obliged to start this dissertation with the standard observation:  there is seemingly limitless diversity to the organisms that have evolved in nature. 
From microbes to megafauna, evolutionary processes have created a stunning array of traits and behaviors in organisms. 
%But how did these features come to exist?
%Were each of these features fated to come to exist, or were their eventual evolution inevitable?
Were each of these features mere evolutionary flukes, or was their eventual evolution inevitable?
%And what
Are there at least general evolutionary trends that we can abstract from these examples? 
Questions like these, of course, are grand challenges of evolutionary biology. 
Here, I focus on one such question: how do the events of the past influence future evolution?


% Plant the seeds of historical contingency
When looking at evolution in nature, we often encounter beneficial traits that were uniquely evolved in one type of organism, while organisms in the broader taxonomic unit found different survival strategies.
For example, most birds and many insects exhibit self-powered flight, but only one branch of mammals does: bats \citep{gunnellFossilEvidenceOrigin2005}. 
Similarly, while many snakes have both venom and the means to deliver it via a bite, venomous bites are only found in one clade of lizards \citep{fryEarlyEvolutionVenom2006}.
%Why was this behavior so rare, and what conditions led to the evolution of flight in only this one specific branch of mammals? 
Why are these adaptations so rare, and what conditions led to their evolution in only a single clade each?
%We can also ask converse questions, such as why a given clade may have lost a seemingly beneficial trait? 
In nature, there is only so much we can do; unfortunately we cannot travel back in time and ``replay the tape of life'' (as evoked by \citet{gouldWonderfulLifeBurgess1990}) to observe if flight consistently evolved in bats thanks to that mammalian clade's history, or if it was an uncommon stroke of luck.

% Dive into historical contingency and overview what I did
While replaying the tape of life \textit{as we know it} is impossible, experimental evolution studies allow us to ask similar questions. 
As I describe below, evolutionary biologists have successfully employed experimental approaches to better understand the role that accumulated history has played in the evolution of microbial traits. 
In this work I continue this endeavor; I use \textit{in silico} evolution experiments to study the role of history in evolving traits, both simple and complex, in digital organisms. 
I first ask how the evolution of one trait, a phenotypically plastic response to the environment, affects downstream evolutionary dynamics. 
I then pivot, switching from observations of how a known trait altered evolution, to retroactively identifying mutations that drastically shifted the long-term fate of the population. 
Finally, I demonstrate how we can leverage these techniques to better understand fundamental evolutionary dynamics. 
Though still in their infancy, these techniques and ideas show amazing potential for understanding the nuanced role that history plays in evolution.