\section{Insights for wet lab studies}

While conducting the research outlined in this dissertation, I have had time to reflect on how the outcomes of these studies -- all of which are digital -- can be transferred to natural organisms and wet lab studies of their evolution. 
This question is, by far, the most common question I receive from evolutionary biologists who do not work with digital organisms themselves. 
As an initial summary, you can consider these digital studies as a rapid form of prototyping and benchmarking. 
These results are valid, and should be considered when trying to understand the role of history in evolution. 
However, my overall goal for my ongoing digital research is two-fold:

First, I wish to refine the methodologies behind techniques such as replay experiments. 
Studies leveraging these techniques are costly; evolving microbial population can take months or years \textit{for a single experiment}. 
By trying these techniques at a larger scale, I hope to identify time-saving optimizations, identify possible pitfalls, and develop meaningful analyses. 
This way, future wet lab studies can conduct similar studies more efficiently and confidently. 
For example, Chapters \ref{chap:learning_case_studies} and \ref{chap:learning_distributions} have demonstrated that a two-phase replay experiment can be used to find potentiating steps while saving substantial amounts of time. 

Second, I hope to provide a baseline that future studies -- be they \textit{in vitro} or \textit{in silico} -- can compare against. 
The need for these benchmark data is evident in this dissertation. 
Upon completion of Chapter \ref{chap:learning_case_studies}, while I was confident in the experimental setup and the evolution system, I was surprised and a mildly concerned by how large the potentiation increases were. 
This result was my original motivation behind conducting the scaled-up experiments for Chapter \ref{chap:learning_distributions}; to ensure that observing such chances was not a probabilistic anomaly. 
%Such large replay studies are unfortunately not currently feasible in wet lab studies, as such studies can take months to only analyze a single lineage.
%This is further demonstrated by the targeted replays in Chapters \ref{chap:learning_case_studies} and \ref{chap:learning_distributions}. 
Unfortunately, conducting replay replicates at the single-mutation resolution for that many replicates, is currently impossible in wet lab systems. 
%I have been able to show that it can be a meaningful experiment if technology allows it in the future. 
As such, this work can serve as the first large-scale benchmark, allowing for comparisons until such large studies are possible to conduct with natural organisms.

Over the course of my PhD, I have shifted from considering myself mainly a ``computer scientist'' to primarily a ``computational evolutionary biologist''. 
With this mindset, the goal in my research has become to facilitate a synergistic interplay between wet lab experimental evolution and digital evolution. 
Wet lab researchers are limited in what experiments they can conduct due to time, money, and experimental control. 
Digital evolution researchers are limited in what their studies can say about evolution in the natural world. 
By capitalizing on the strengths of each discipline, we can continue to cover the challenges of the other. 
%This cross-discipline interchange is my goal, and the mindset with which this dissertation was written. 