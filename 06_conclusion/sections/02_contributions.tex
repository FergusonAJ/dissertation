\section{Contributions}

Here I summarize the contributions of each research chapter. 

In \textbf{Chapter \ref{chap:consequences_of_plasticity}} I experimentally measured the impact of evolved phenotypic plasticity on further evolution. 
I used the digital evolution software Avida to evolve both plastic and non-plastic populations in a cyclically changing environment. 
Once evolved, I moved populations to three novel environments to observe the difference that plasticity made in each. 
First, I found that phenotypic plasticity slows the rate of evolutionary change compared to the non-plastic populations that must continuously adapt to environmental changes. 
Second, I showed that plastic populations accumulated novel beneficial traits at the same rate as non-plastic populations, but plastic populations were able to retain the traits while non-plastic populations continuously lost them. 
Finally, I examined how, in their continuous struggle to adapt to the environment, non-plastic populations accumulate deleterious traits significantly more often than plastic populations. 
Overall this chapter identified how the evolution of one trait, phenotypic plasticity, shifts the dynamics of further evolution by drastically reducing the impact of environmental changes. 

Next, in \textbf{Chapter \ref{chap:learning_case_studies}}, I quantified how the likelihood of evolving associative learning (i.e., the potentiation of associative learning) changed over the course of four case study lineages. 
To do this, I employed ``analytical replay experiments'', restarting evolution from various points along each lineage. 
Across the four lineages, I found substantial single-step increases in potentiation, ranging from 36 to 64 percentage points. 
I then analyzed the individual mutations that increased potentiation, finding a variety of dynamics in these potentiating mutations. 
In this chapter, I demonstrated that digital evolution can be used to perform studies on potentiation dynamics that are intractable in living organisms.
In this case I used replay experiments to track down individual potentiating mutations.  

I then extended this work in \textbf{Chapter \ref{chap:learning_distributions}}, analyzing the potentiation dynamics of 50 new lineages that were able to evolve associative learning.  
I found that the large single-step potentiation increases in Chapter \ref{chap:learning_case_studies} were not a fluke. 
While the distribution of all observed single-step potentiation increases was centered around zero, the maximum per-replicate increase was significant for all 50 replicates. 
Indeed, looking at the distribution of maximum per-step potentiation increase across the 50 replicates, I found a range of [20, 88], a mean of approximately 42, and a median of 38 percentage points. 
I also replayed ten lineages that did not evolve learning, and in each I found significant single-step increases in the potentiation of the most likely behavior at the end of evolution. 
Finally, I found that in 44 of 50 replicates the largest potentiation gain was caused by a single mutation, and all but one of these mutations were either beneficial or neutral, with 15 mutations causing no change to the organism's phenotype. 

\textbf{Chapter \ref{chap:adaptive_momentum}} pivots to using replay experiments and potentiation to illustrate the adaptive momentum effect in a simple digital model. 
The adaptive momentum framework posits that disequilibrium, such as a selective sweep, can temporarily increase evolutionary exploration. 
By quantifying the potentiation of crossing a deleterious fitness valley, I show that early mutations into the valley drastically increase potentiation when the population is in disequilibrium, yet have no detectable effect when the population is in equilibrium. 
By performing replays with the organisms in the population shuffled at random before starting, I show that population structure plays a critical role in this scenario. 
This work also demonstrates how replaying population snapshots instead of clonal populations can improve the granularity of the potentiation dynamics, and in fact is especially relevant to the dynamics of adaptive momentum. 

All together, these chapters 
1) advance our understanding of the role that history plays in evolution
2) showcase the power of replay experiments in finding critical points in a population's history and 
3) demonstrate the advantage of using digital models to test and refine methodologies in evolutionary biology before applying them to natural organisms in the lab. 