%While it may seem obvious that history shapes how a population evolves, 
%In this dissertation I have shown \textit{how} historical contingency can shape long-term evolutionary outcomes. 
This dissertation constitutes an important step in understanding \textit{how} historical contingency can shape long-term evolutionary outcomes.
Using digital evolution, I have shown that the evolution of phenotypic plasticity can act as an evolutionary stabilizer, negating the effects of environmental variation on future evolutionary dynamics.
By leveraging analytic replay experiments, I have shown that a single mutation can shift the likelihood of an evolutionary outcome from incredibly unlikely to almost certain. 
Further, I have demonstrated that the same replay techniques can be used to clearly illustrate basic evolutionary dynamics via illuminating the long-term impacts of short-term changes. 

There remains an enormous amount work to fully understand the role of history in evolution. 
Thankfully, however, this work has given me some insight into how we think about historical contingency, potentiation, and replay experiments, as well as the future work that should be done to further our understanding. 
This chapter is dedicated to recapitulating the lessons learned throughout this dissertation and discussing how we can build upon them to continue enhancing our understanding of historical contingency in evolution. 
